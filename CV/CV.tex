%----------------------------------------------------------------------------------------
%       Evan J. Arena
%       Curriculum Vitae
%       Last updated: 15 October 2024
%----------------------------------------------------------------------------------------
%	PACKAGES AND OTHER DOCUMENT CONFIGURATIONS
%----------------------------------------------------------------------------------------

\documentclass{resume} % Use the custom resume.cls style

\usepackage[left=0.75in,top=0.6in,right=0.75in,bottom=0.6in]{geometry} % Document margins
\usepackage{hyperref}
\usepackage[usenames,dvipsnames]{xcolor}
\hypersetup{
    colorlinks = true,
    citecolor = {MidnightBlue},
    linkcolor = {BrickRed},
    urlcolor = {BrickRed}
}
\usepackage{enumitem}
\usepackage{etaremune}

%\usepackage{bibentry}

\newcommand{\forceindent}{\leavevmode{\parindent=1em\indent}}

\name{Evan J. Arena} % Name


\address{Park Science Center, Office \textnumero \,130  \\ 101 N. Merion Avenue \\ Bryn Mawr, PA 19010, USA} % Address
\address{+1~$\cdot$~(610)~$\cdot$~526~$\cdot$~ext. 7561 \\ \href{mailto:earena@brynmawr.edu}{earena@brynmawr.edu} \\ \href{https://evanjarena.github.io}{https://evanjarena.github.io}} % Phone number and email
\address{Lecturer and Lab Coordinator \\ Deptartment of Physics \\ Bryn Mawr College } % Title

\pagenumbering{roman}

\begin{document}

  %\nobibliography{CV.bib}
  %\bibliographystyle{unsrt}

%----------------------------------------------------------------------------------------
%	RESEARCH INTERESTS SECTION
%----------------------------------------------------------------------------------------

\begin{rSection}{Research Interests}

\textbf{Theoretical astrophysics and cosmology}, including general relativity, gravitational lensing, modified gravity, large-scale structure, 21\,cm cosmology, dark energy, inflation, dark matter, radio astronomy, and gravitational waves. 

\end{rSection}

%----------------------------------------------------------------------------------------
%	EDUCATION SECTION
%----------------------------------------------------------------------------------------

\begin{rSection}{Education}

\textbf{Drexel University} \\%\hfill {2018 -- Present} \\ 
{\color{MidnightBlue} Ph.D.} in Physics \hfill {June 2024} \\
\textit{Thesis: ``Flexion in gravitation and cosmology''}\\
\textit{Advisor: David M. Goldberg, Ph.D., Co-advisor: David. J. Bacon, Ph.D.}\\
{\color{MidnightBlue} M.S.} in Physics \hfill{June 2020}%\\%\textit{GPA: 3.95}

\textbf{Stony Brook University} \\%\hfill {2013 -- 2017} \\
{\color{MidnightBlue} B.S.} in Physics, second major: Astronomy/Planetary Sciences \hfill{May 2017}\\%\textit{GPA: 3.55 (Cum Laude)}\\
\textit{Cum Laude}\\
\textit{Departmental Honors in Physics}

\end{rSection}

%----------------------------------------------------------------------------------------
%	POSITIONS HELD SECTION
%----------------------------------------------------------------------------------------

\begin{rSection}{Positions Held}

\textbf{Bryn Mawr College} \hfill 2024 -- Present \\
\textit{Lecturer of Physics and Lab Coordinator} \\
Department of Physics
  
\textbf{Drexel University} \hfill {2018 -- 2024} \\
\textit{Doctoral Research Fellow, Doctoral Teaching Fellow, And CoAS Dean's Fellow}\\
Department of Physics

\textbf{Stony Brook University and Brookhaven National Laboratory} \hfill {2015 -- 2019}\\
\textit{Research Assistant}\\
SBU Department of Physics \& Astronomy  and BNL Department of Physics

\textbf{Brookhaven National Laboratory} \hfill {2012 -- 2013}\\
\textit{Intern}\\
Department of Physics

\end{rSection}

%----------------------------------------------------------------------------------------
%	GRANTS SECTION
%----------------------------------------------------------------------------------------

\begin{rSection}{Funded Grants}

  \textbf{National Science Foundation (NSF) \href{https://www.nsf.gov/awardsearch/showAward?AWD_ID=2427091}{Grant \textnumero 2427091}} \textbf{\textit{``ExpandQISE:}} \hfill {2024 -- 2029} \\
\textbf{\textit{Track 2:  Research and Education Center for Quantum Materials and\\ Sensing  at a  Women's College''}}\\ 
\textit{Co-PI}\\
NSF organization: Office of Strategic Initiatives\\
Award amount: \$5,000,000
  
\textbf{National Science Foundation (NSF) \href{https://www.nsf.gov/awardsearch/showAward?AWD_ID=2306989&HistoricalAwards=false}{Grant \textnumero 2306989} \textit{``Cosmic Flexion''}} \hfill {2023 -- 2026} \\
\textit{Funded Graduate Student}\\
NSF organization: Division of Astronomical Sciences\\
NSF program: Extragalactic Astronomy \& Cosmology\\
Award amount: \$359,436.00\\
Contribution: Wrote significant portion of proposal.

\end{rSection}

%\newpage

%----------------------------------------------------------------------------------------
%	AWARDS AND HONORS SECTION
%----------------------------------------------------------------------------------------

\begin{rSection}{Awards and Honors}

 \textit{Graduate College Continuing Excellence in Teaching Assistance Award}, Drexel University \hfill{2023} \\
\textit{Graduate College Continuing Excellence in Teaching Assistance Award}, Drexel University \hfill{2022} \\
\textit{Graduate College Continuing Excellence in Teaching Assistance Award}, Drexel University \hfill{2021} \\
\textit{Graduate College Teaching Assistant Excellence Award}, Drexel University \hfill{2020} \\
\textit{Sigma Xi Scientific Research Honor Society Member}, Drexel University \hfill{2019}\\
\textit{College of Arts and Sciences (CoAS) Dean's Fellowship}, Drexel University \hfill {2018}\\
\textit{Sigma Pi Sigma National Physics Honor Society Member}, Stony Brook University \hfill{2017}\\
\textit{Presidential Scholarship}, Stony Brook University \hfill {2013}

\end{rSection}

%----------------------------------------------------------------------------------------
%	RESEARCH HISTORY SECTION
%----------------------------------------------------------------------------------------

\begin{rSection}{Research History}

\begin{description}[leftmargin=8em, style=nextline]

\item[\textnormal{2018 -- Present}] \textbf{Weak gravitational lensing}\\
   Developed a novel method for measuring the second-order weak gravitational lensing effect known as flexion;
  Created a full theoretical formalism for ``cosmic flexion'' -- a family of cosmological weak lensing signals originating from the large-scale structure of the universe;  Discovered previously unknown cosmological weak lensing signals and posited the existence of non-commutativity in weak lensing;
  Measurement of flexion in the Dark Energy Survey, including building the largest flexion catalogue to date as well as making the first ever detection of cosmic flexion;
  Discovered unique weak lensing signatures for negative mass compact objects and exotic objects such as the Ellis wormhole.
 
\item[\textnormal{2015 -- 2019}] \textbf{Low redshift 21$\,$cm intensity mapping}\\
 Cosmological parameter and modified gravity forecasts for a general 21$\,$cm cosmology 
 experiment, member of the DOE Cosmic Visions Dark Energy 21$\,$cm Working Group, and design and construction
 of the radio telescope used for the 21$\,$cm Baryon Mapping eXperiment at Brookhaven National
 Laboratory.
\item[\textnormal{2013}] \textbf{Gravitational waves}\\
  New method for the indirect detection of gravitational waves.
  %Proposed a new method for the indirect detection of gravitational waves via precision stellar redshift measurement.
  
\item[\textnormal{2012}] \textbf{Modified Newtonian Dynamics}\\
 Investigated the plausibility of Modified Newtonian Dynamics on a local scale based on
 rotation curves of the Milky Way.

\end{description}

\end{rSection}

%----------------------------------------------------------------------------------------
%	REFEREED PUBLICATIONS SECTION
%----------------------------------------------------------------------------------------

\begin{rSection}{Refereed Publications}

\begin{etaremune}

\item \textbf{{Arena}, E.~J.}, \textit{``Weak gravitational flexion in various spacetimes: Exotic lenses and modified gravity,"} Phys.Rev.D \textbf{106}, 064019 (2022) \href{https://arxiv.org/abs/2207.07784}{[arXiv:2207.07784]}
    
\item \textbf{{Arena}, E.~J.}, {Goldberg}, D.~M., and {Bacon}, D.~J., \textit{``Cosmic flexion,"} Phys.Rev.D \textbf{105}, 123521 (2022) \href{https://arxiv.org/abs/2203.12036}{[arXiv:2203.12036]}
  
\item {Fabritius}, J.~M., \textbf{{Arena}, E.~J.}, and {Goldberg}, D.~M. \textit{``Shape, color, and distance in weak gravitational flexion,"} Mon.Not.Roy.Astron.Soc. \textbf{501}, 4103 (2021) \href{https://arxiv.org/abs/2006.03506}{[arXiv:2006.03506]}
\end{etaremune}

In preparation:
\begin{etaremune}
\item \textbf{{Arena}, E.~J.}, {Goldberg}, D.~M., {Bacon}, D.~J., and the Dark Energy Survey Collaboration, \textit{``Evidence for cosmic flexion in the Dark Energy Survey Year 3 data,"} in preparation.
 \end{etaremune}

\end{rSection}

%----------------------------------------------------------------------------------------
%	CONFERENCE PROCEEDINGS, SCIENCE BOOKS, WHITE PAPERS SECTION
%----------------------------------------------------------------------------------------

\newpage

\begin{rSection}{Conference Proceedings, Science Books, White Papers}

\begin{etaremune}

\item {Timbie}, P. et al., including \textbf{{Arena}, E.~J.},
%\item {Ahmed}, Z., {Alonso}, D., {Amin}, M.~A.,
%         {Ansari}, R., \textbf{{Arena}, E.~J.}, {Bandura}, K.,
%         {Beardsley}, A., {Bull}, P., {Castorina}, E.,
%         {Chang}, T.-C., {Dav{\'e}}, R., {Dillon}, J.~S.,
%         {van Engelen}, A., {Ewall-Wice}, A., {Ferraro}, S.,
%         {Foreman}, S., {Frisch}, J., {Green}, D.,
%         {Holder}, G., {Jacobs}, D., {Karagiannis}, D.,
%         {Kaurov}, A.~A., {Knox}, L., {Kuhn}, E.,
%         {Liu}, A., {Ma}, Y.-Z., {Masui}, K.~W.,
%         {McClintock}, T., {Moodley}, K.,
%         {M{\"u}nchmeyer}, M., {Newburgh}, L.~B.,
%         {Nomerotski}, A., {O'Connor}, P., {Obuljen}, A.,
%         {Padmanabhan}, H., {Parkinson}, D., {Perdereau}, O.,
%         {Rapetti}, D., {Saliwanchik}, B., {Sehgal}, N.,
%         {Shaw}, J.~R., {Sheehy}, C., {Sheldon}, E.,
%         {Shirley}, R., {Silverstein}, E., {Slatyer}, T.,
%         {Slosar}, A., {Stankus}, P., {Stebbins}, A.,
%         {Timbie}, P., {Tucker}, G.~S., {Tyndall}, W.,
%         {Villaescusa-Navarro}, F., and {Wulf}, D.,
         \textit{``Research and Development for HI Intensity Mapping,''} ArXiv e-prints (2019) \href{https://arxiv.org/abs/1907.13090}{[arXiv:1907.13090]}

\item {Slosar}, A. et al., including \textbf{{Arena}, E.~J.},
%\item {Ahmed}, Z., {Alonso}, D., {Amin}, M.~A.,
%         {Ansari}, R., \textbf{{Arena}, E.~J.}, {Bandura}, K.,
%         {Battaglia}, N, {Blazek}, J., 
%         {Bull}, P., {Castorina}, E.,
%         {Chang}, T.-C., {Connor}, L., {Dav{\'e}}, R., {Dillon}, J.~S.,
%         {Dvorkin}, C.,
%         {van Engelen}, A., {Ferraro}, S., {Flauger}, R.,
%         {Foreman}, S., {Frisch}, J., {Green}, D.,
%         {Holder}, G., {Jacobs}, D., {Johnson}, M.~C.,
%         {Karagiannis}, D.,
%         {Kaurov}, A.~A., {Knox}, L., 
%         {Liu}, A., {Loverde}, M., {Ma}, Y.-Z., {Masui}, K.~W.,
%         {McClintock}, T., {Meerburg}, P.~D., {Moodley}, K.,
%         {M{\"u}nchmeyer}, M., {Newburgh}, L.~B., {Ng}, C.,
%         {Nomerotski}, A., {O'Connor}, P., {Obuljen}, A.,
%         {Padmanabhan}, H., {Parkinson}, D., {Prochaska}, J.~X.,
%         {Rajendran}, S.,
%         {Rapetti}, D., {Saliwanchik}, B., {Schaan}, E., {Sehgal}, N.,
%         {Shaw}, J.~R., {Sheehy}, C., {Sheldon}, E.,
%         {Shirley}, R., {Silverstein}, E., {Slatyer}, T.,
%         {Slosar}, A., {Stankus}, P., {Stebbins}, A.,
%         {Timbie}, P., {Tucker}, G.~S., {Tyndall}, W.,
%         {Villaescusa-Navarro}, F., {Wallisch}, B., and {White}, M.,
\textit{``Packed Ultra-wideband Mapping Array (PUMA): A Radio Telescope for Cosmology and Transients,''}, Bull.Am.Astron.Soc. \textbf{51}, 53 (2019)  \href{https://arxiv.org/abs/1907.12559}{[arXiv:1907.12559]}

\item {Cosmic Visions 21$\,$cm Collaboration}, including \textbf{{Arena}, E.~J.},
%\item {Cosmic Visions 21$\,$cm Collaboration}, {Ansari}, R., \textbf{{Arena}, E.~J.} , 
%	{Bandura}, K., {Bull}, P., {Castorina}, E., {Chang}, T.-C., 
%	{Foreman}, S., {Frisch}, J., {Green}, D., {Karagiannis}, D., 
%	{Liu}, A., {Masui}, K.~W., {Meerburg}, P.~D., {Newburgh}, L.~B., 
%	{Obuljen}, A., {O'Connor}, P., {Shaw}, J.~R., {Sheehy}, C., 
%	{Slosar}, A., {Smith}, K., {Stankus}, P., {Stebbins}, A., 
%	{Timbie}, P., {Villaescusa-Navarro}, F., and {White}, M., 
  \textit{``Inflation and Early Dark Energy with a {Stage~{\sc ii}} Hydrogen Intensity Mapping experiment,''} ArXiv e-prints (2018) \href{https://arxiv.org/abs/1810.09572}{[arXiv:1810.09572]}


\end{etaremune}

\end{rSection}

%----------------------------------------------------------------------------------------
%	TALKS AND PRESENTATIONS SECTION
%----------------------------------------------------------------------------------------
%\newpage %Just for now until more pubs. are added

\begin{rSection}{Conferences and Talks}
\textbf{Invited Talks}
\begin{etaremune}
\item Astro Lunch Seminar at the University of Sussex; \textit{``Constraining the dark universe with light bananas;''} Falmer, East Sussex, United Kingdom; 16 Nov. 2023
\item Colloquium at the Institute of Cosmology and Gravitation, University of Portsmouth; \textit{``Constraining the dark universe with light bananas;''} Portsmouth, Hampshire, United Kingdom; 9 Nov. 2023
\end{etaremune}

\textbf{Contributed Talks}
\begin{etaremune}
\item Dark Energy Survey Fall Collaboration Meeting; \textit{ ``The DES Y3 Weak Lensing Flexion Catalogue;''} NCSA at UI Urbana-Champaign; Urbana-Champaign, Illinois, USA; 10 Oct. 2023
\item AstroPhilly `23; \textit{``Constraining the small-scale matter power spectrum with cosmic flexion;''} Villanova University; Villanova, Pennsylvania, USA; 27 July 2023
\item Talk to DES Weak Lensing Working Group; \textit{``Weak gravitational flexion in the Dark Energy Survey;''} Virtual Meeting; 11 May 2022
\item Research talk to incoming graduate students; \textit{``Hybrid analytic image modeling and image moments approach to gravitational lensing;''} Drexel University; Philadelphia, Pennsylvania, USA; 17 Sep. 2019
 \item  High School Research Program conference; \textit{``Observation of gravitational waves through precision stellar redshift measurement;''} Brookhaven National Laboratory; Brookhaven, New York, USA;  16 Aug. 2013
\end{etaremune}

\textbf{Poster Presentations}
\begin{etaremune}
\item  First-year graduate student presentations; \textit{``Hybrid analytic image modeling and image moments approach to gravitational lensing;''} Drexel University; Philadelphia, Pennsylvania, USA; 11 Jun. 2019
\item High School Research Program conference; \textit{``Dark matter and its alternatives;''} Brookhaven National Laboratory; Brookhaven, New York, USA; 27 Nov. 2012
\end{etaremune}
  
\end{rSection}

%----------------------------------------------------------------------------------------
%	SOFTWARE SECTION
%----------------------------------------------------------------------------------------
%\newpage

\begin{rSection}{Software Developed}

\underline{\makebox[0.965\textwidth][l]{\textbf{Authored}}}

\begin{description}[leftmargin=10em, style=nextline]

\item[\texttt{F-SHARP}] Code for computing weak gravitational lensing correlations. \textit{Publicly available code written in Python}. \href{https://github.com/evanjarena/F-SHARP}{https://github.com/evanjarena/F-SHARP}

\item[\texttt{Lenser}] A tool for measuring weak gravitational flexion. \textit{Publicly available code written in Python}. \href{https://github.com/DrexelLenser/Lenser}{https://github.com/DrexelLenser/Lenser}

\item[21cmMG] A suite for probing modified gravity with 21\,cm cosmology. \textit{Publicly available code written in Python}. \href{https://github.com/evanjarena/21cmMG}{https://github.com/evanjarena/21cmMG}

\item[Fisher21cm] Fisher forecast for a general 21\,cm experiment. \textit{Publicly available code written in Python}. \href{https://github.com/evanjarena/Fisher21cm}{https://github.com/evanjarena/Fisher21cm}

\end{description}

\underline{\makebox[0.965\textwidth][l]{\textbf{Contributed}}}

\begin{description}[leftmargin=10em, style=nextline]

\item[\texttt{PythonOpenMPI}] A generalizable utility for efficient task-based parallel programming using the \texttt{mpi4py} library. \textit{Publicly available code written in Python}. \\ \href{https://github.com/seanlabean/PythonOpenMPI}{https://github.com/seanlabean/PythonOpenMPI}

\item[\texttt{LensTools}] Useful computing tools for weak lensing analyses. \textit{Publicly available code written in Python}. \href{https://github.com/apetri/LensTools}{https://github.com/apetri/LensTools}

\end{description}

\end{rSection}

%----------------------------------------------------------------------------------------
%	TEACHING SECTION
%----------------------------------------------------------------------------------------


\begin{rSection}{Teaching}

\textbf{Bryn Mawr College} \\
\textit{Lecturer}\\
\forceindent PHYS B101, \textit{Introductory Physics I} \hfill{Fall 2024}
\begin{description}[leftmargin=2em, style=nextline]
\vspace{-0.5em}
\item[~] \textit{This is the first of two courses in the introductory physics sequence intended primarily for students on the pre-health professions track. Emphasis is on developing an understanding of how we study the universe, the ideas that have arisen from that study, and on problem solving. Topics are taken from among Newtonian kinematics and dynamics, relativity, gravitation, and fluid mechanics. An effective and usable understanding of algebra and trigonometry is assumed.}\\
\forceindent\forceindent F`24: Undergraduate lecture section (with accompanying recitation), 37 students
\end{description}
\textit{Lab Coordinator}\\
\forceindent PHYS B101 Lab, \textit{Introductory Physics I Laboratory} \hfill{Fall 2024}
\begin{description}[leftmargin=2em, style=nextline]
\vspace{-0.5em}
\item[~] \textit{Lab run independently from parent course.}\\
\forceindent\forceindent F`24: Undergraduates and Post-baccalaureates, 97 students
\end{description} 
\textbf{Drexel University} \\ 
%\underline{\makebox[0.965\textwidth][l]{\textit{Teaching Assistant} (Recitation and Lab Instructor)}}\\
\textit{Teaching Assistant} (Recitation and Lab Instructor)\\
\forceindent PHYS 100, \textit{Preparation for Engineering Studies}\hfill {Winter: 2023, 2021, 2020, 2019}
\begin{description}[leftmargin=2em, style=nextline]
\vspace{-0.5em}
\item[~] \textit{This is a basic mathematics foundational course to prepare the students for the beginning sequence of Engineering Physics. Topics include (but are not limited to): linear and quadratic equations, simultaneous equations, basic geometry, use of trigonometric functions, vectors, translational kinematics, and Newton’s Laws.} \\
\forceindent \forceindent W`23: 3 recitation sections, 65 students total\\
\forceindent \forceindent W`21: 3 recitation sections, 63 students total\\
\forceindent \forceindent W`20: 4 recitation sections, 105 students total\\
\forceindent \forceindent W`19: 3 recitation sections, 86 students total
\end{description}
\vspace{-0.5em}
\forceindent PHYS 152, \textit{Introductory Physics I}  \hfill {Spring: 2023, 2022, 2021, 2020, 2019}
\begin{description}[leftmargin=2em, style=nextline]
\vspace{-0.5em}
\item[~]\textit{This class is the first part of a three-course algebra-based sequence that provides a comprehensive introduction to physics and covers the fundamentals of mechanics. Topics include motion in one or more dimensions, Newton’s laws, gravitation, energy, momentum, and rotational motion. This course includes in-person labs that are intended to enrich the concepts presented in lecture and recitation section.} \\
\forceindent \forceindent S`23: 3 recitation sections, 43 students total\\
\forceindent \forceindent S`22: 3 recitation sections, 50 students total\\
\forceindent \forceindent S`21: 4 recitation section, 87 students total\\
\forceindent \forceindent S`20: 1 recitation section, 70 students total\\
\forceindent \forceindent S`19: 4 recitation sections, 70 students total
\end{description}
\vspace{-0.5em}
\forceindent PHYS 154, \textit{Introductory Physics III}  \hfill {Fall: 2022, 2021, 2020, 2019, 2018}
\begin{description}[leftmargin=2em, style=nextline]
\vspace{-0.5em}
\item[~]\textit{This class is the third part of a three-course algebra-based sequence providing a comprehensive introduction to physics and covers the fundamentals of electricity and magnetism. Topics include electric charges, electric fields, electric potential, DC circuits, magnetic induction, electromagnetic waves, special relativity, and optical interference. This course includes labs that are intended to enrich the concepts presented in lecture and recitation section.} \\
\forceindent \forceindent F`22: 3 recitation sections, 64 students total\\
\forceindent \forceindent F`21: 3 recitation sections, 58 students total\\
\forceindent \forceindent F`20: 2 recitation sections and 1 lab section, 84 students total\\
\forceindent \forceindent F`19: 4 recitation sections, 92 students total\\
\forceindent \forceindent F`18: 1 recitation section and 1 lab section, 42 students total
\end{description}
\vspace{-0.5em}
\textit{Grader} \\
\forceindent PHYS 131, \textit{Survey of the Universe} \hfill {Winter 2022}
\begin{description}[leftmargin=2em, style=nextline]
\vspace{-0.5em}
\item[~]\textit{This is a three-credit elective course that provides an overview of modern astronomy including the scientific method, telescopes, stars and star clusters, stellar evolution, galaxies and the large-scale structure of the universe, and the Big Bang. The online version of this course is designed to engage students in an investigation of astronomy in a more active way; the hope is that, with this interactive video game platform, students will achieve a greater understanding and appreciation of astronomy.}
\end{description}
\vspace{-0.5}
\forceindent PHYS 231, \textit{Introductory Astrophysics} \hfill {Winter 2022}\\
\textit{Guest Lecturer} \\
\forceindent PHYS 231, \textit{Introductory Astrophysics} \hfill {Winter 2022}
\begin{description}[leftmargin=2em, style=nextline]
\vspace{-0.5em}
\item[~]\textit{This is an introductory astrophysics course aimed for science majors. Topics include
a treatment of orbits, Kepler’s laws, celestial coordinates, light, blackbodies, optics, stellar structure and
evolution, galactic formation, and large scale evolution and structure of the universe.}\\
\forceindent \forceindent W`21: 1 Lecture, 25 students total
\end{description}
\vspace{-0.5}
\textbf{Stony Brook University} \\
\textit{Lecturer}\\
\forceindent  Della Pietra High School Applied Math Program \hfill {Spring 2017}

%\textbf{Drexel University} \\ 
%\textit{Teaching Assistant} (Recitation and Lab Instructor)\\
%\forceindent PHYS 100, \textit{Preparation for Engineering Studies}  \hfill {Winter: 2023, 2021, 2020, 2019}\\
%\forceindent PHYS 152, \textit{Introductory Physics I}  \hfill {Spring: 2023, 2022, 2021, 2020, 2019}\\
%\forceindent PHYS 154, \textit{Introductory Physics III}  \hfill {Fall: 2022, 2021, 2020, 2019, 2018}\\
%\textit{Grader} \\
%\forceindent PHYS 131, \textit{Survey of the Universe} \hfill {Winter 2022}\\
%\forceindent PHYS 231, \textit{Introductory Astrophysics} \hfill {Winter 2022}\\
%\textit{Guest Lecturer} \\
%\forceindent PHYS 231, \textit{Introductory Astrophysics} \hfill {Winter 2022}\\
%\textbf{Stony Brook University} \\
%\textit{Lecturer}\\
%\forceindent  Della Pietra High School Applied Math Program \hfill {Spring 2017}

\end{rSection}

%----------------------------------------------------------------------------------------
%	PROFESSIONAL ACTIVITIES AND SERVICE SECTION
%----------------------------------------------------------------------------------------

%\newpage %Just for now

\begin{rSection}{Professional Activities and Service}

\begin{description}[leftmargin=10em, style=nextline]

\item[Collaborations] External Collaborator, Dark Energy Survey (DES)\\
  Member, Packed Ultra-wideband Mapping Array (PUMA) [Inactive]\\
  Member, Baryon Mapping eXperiment (BMX) [Inactive]

\item[Working Groups] Member, DOE Cosmic Visions Dark Energy 21$\,$cm Working Group [Inactive]

%\item[Collaborations] Inactive member of the Large Synoptic Survey Telescope Dark Energy 
% Science Collaboration (LSST-DESC)

\end{description}
\textbf{Media Appearances}\\
Appeared on \textit{Good Day Philadelphia} on FOX Philadelphia 29 to talk about Earth's ``second moon'' as well as a glacial landslide in Greenland and how it relates to climate change: \\
\href{https://www.fox29.com/video/1523179}{https://www.fox29.com/video/1523179} (27 Sep. 2024).

Appeared on the \textit{Anthony Gargano Show} to discuss the solar eclipse viewing from Philadelphia: \href{https://www.youtube.com/watch?v=gkbAUbbIH20}{https://www.youtube.com/watch?v=gkbAUbbIH20} (8 Apr. 2024).

\textbf{Outreach Activities and Service}\\
\textit{Bryn Mawr College}
\begin{description}[leftmargin=2em, style=nextline]
\vspace{-0.5em}
\item[~] Ran a pedagogy workshop for first-year graduate TAs in the Bryn Mawr Graduate School of Arts and Sciences, titled \textit{``Becoming an effective TA: Pedagogies to support learning''} (28 Aug. 2024).
\end{description}


\textit{Drexel University}
\begin{description}[leftmargin=2em, style=nextline]
\vspace{-0.5em}
\item[~] Organized and ran a public solar eclipse viewing at Drexel University (8 Apr. 2024).
\item[~]Helped run the Kaczmarczik Lecture Series Open House, hosted by the Drexel University Department of Physics (7 Mar. 2024).
\item[~]Free physics tutoring at the Stony Brook University Veterans Student Organization (2023).
\item[~]Assist in running the monthly Drexel Physics Department open house, where we open the the Joseph R. Lynch Observatory for public viewing (2018 -- Present).
\item[~]Invited to appear on the Drexel University Teaching Assistant Orientation Panel, as part of the Teaching Assistant Orientation and Preparation Course GRAD T580 (17 Sep. 2020).
\item[~]Helped run the Kaczmarczik Lecture Series Open House, hosted by the Drexel University Department of Physics (14 Nov. 2018).
\end{description}

\textbf{Committee Work}\\
\textit{Bryn Mawr College}
\begin{description}[leftmargin=2em, style=nextline]
\vspace{-0.5em}
\item[~] Tenure Track Faculty Search Committee for the Physics Department (2024)
\end{description}

\textit{Drexel University}
\begin{description}[leftmargin=2em, style=nextline]
\vspace{-0.5em}
\item[~] Treasurer of the Drexel University Physics Graduate Student Association (2020 -- 2021).
\end{description}


\end{rSection}


%----------------------------------------------------------------------------------------
%	TECHNICAL SKILLS SECTION
%----------------------------------------------------------------------------------------

\begin{rSection}{Technical Skills}

Proficient in \textit{Python}.\\
Proficient in Bash and Linux environments.\\
Extensive experience with supercomputing clusters and performing parallel computation.

\end{rSection}


\end{document}
