%----------------------------------------------------------------------------------------
%       Evan J. Arena
%       Curriculum Vitae
%       Last updated: 17 April 2018
%----------------------------------------------------------------------------------------
%	PACKAGES AND OTHER DOCUMENT CONFIGURATIONS
%----------------------------------------------------------------------------------------

\documentclass{resume} % Use the custom resume.cls style

\usepackage[left=0.75in,top=0.6in,right=0.75in,bottom=0.6in]{geometry} % Document margins
\usepackage{hyperref}
\usepackage[usenames,dvipsnames]{xcolor}
\hypersetup{
    colorlinks = true,
    citecolor = {MidnightBlue},
    linkcolor = {BrickRed},
    urlcolor = {BrickRed}
}
\usepackage{enumitem}
\usepackage{etaremune}

%\usepackage{bibentry}

\newcommand{\forceindent}{\leavevmode{\parindent=1em\indent}}

\name{Evan J. Arena} % Name


\address{Disque Hall, Office No. 808 \\ 32 S. 32$^{\rm nd}$ St. \\ Philadelphia, PA 19104, USA} % Address
\address{+1~$\cdot$~(516)~$\cdot$~383~$\cdot$~4817 \\ \href{mailto:evan.james.arena@drexel.edu}{evan.james.arena@drexel.edu}} % Phone number and email
\address{Ph.D. Candidate \\ Deptartment of Physics \\ Drexel University } % Title

\pagenumbering{roman}

\begin{document}

  %\nobibliography{CV.bib}
  %\bibliographystyle{unsrt}

%----------------------------------------------------------------------------------------
%	RESEARCH INTERESTS SECTION
%----------------------------------------------------------------------------------------

\begin{rSection}{Research Interests}

\textbf{Theoretical astrophysics and cosmology}, including general relativity, gravitational lensing, modified gravity, large-scale structure, 21\,cm cosmology, dark energy, inflation, dark matter, radio astronomy, and gravitational waves. 

\end{rSection}

%----------------------------------------------------------------------------------------
%	EDUCATION SECTION
%----------------------------------------------------------------------------------------

\begin{rSection}{Education}

\textbf{Drexel University} \\%\hfill {2018 -- Present} \\ 
{\color{MidnightBlue} Ph.D.} Student/Candidate of Physics \hfill {2018 -- Present} \\
{\color{MidnightBlue} M.S.} in Physics \hfill{2020}\\
\textit{GPA: 3.95}

\textbf{Stony Brook University} \\%\hfill {2013 -- 2017} \\
{\color{MidnightBlue} B.S.} in Physics and Astronomy/Planetary Sciences \hfill{2017}\\
\textit{Cum Laude}\\
\textit{Departmental Honors in Physics}

\end{rSection}

%----------------------------------------------------------------------------------------
%	POSITIONS HELD SECTION
%----------------------------------------------------------------------------------------

\begin{rSection}{Positions Held}

\textbf{Drexel University} \hfill {2018 -- Present} \\
\textit{Doctoral Teaching Fellow and CoAS Dean's Fellow}\\
Department of Physics

\textbf{Stony Brook University and Brookhaven National Laboratory} \hfill {2015 -- 2019}\\
\textit{Research Assistant}\\
SBU Department of Physics \& Astronomy  and BNL Department of Physics

\textbf{Brookhaven National Laboratory} \hfill {2012 -- 2013}\\
\textit{Intern}\\
Department of Physics

\end{rSection}

%----------------------------------------------------------------------------------------
%	AWARDS AND HONORS SECTION
%----------------------------------------------------------------------------------------

\begin{rSection}{Awards and Honors}

\textit{Graduate College Continuing Excellence in Teaching Assistance Award}, Drexel University \hfill{2021} \\
\textit{Graduate College Teaching Assistant Excellence Award}, Drexel University \hfill{2020} \\
\textit{Sigma Xi Scientific Research Honor Society Member}, Drexel University \hfill{2019}\\
\textit{College of Arts and Sciences (CoAS) Dean's Fellowship}, Drexel University \hfill {2018}\\
\textit{Sigma Pi Sigma National Physics Honor Society Member}, Stony Brook University \hfill{2017}\\
\textit{Presidential Scholarship}, Stony Brook University \hfill {2013}

\end{rSection}

%----------------------------------------------------------------------------------------
%	RESEARCH HISTORY SECTION
%----------------------------------------------------------------------------------------

\begin{rSection}{Research History}

\begin{description}[leftmargin=8em, style=nextline]

\item[\textnormal{2018 -- Present}] \textbf{Gravitational Lensing}\\
Developing a novel method for measuring the second-order weak gravitational lensing effect known as flexion. 
\item[\textnormal{2015 -- Present}] \textbf{Low redshift 21$\,$cm intensity mapping}\\
 Cosmological parameter and modified gravity forecasts for a general 21$\,$cm cosmology 
 experiment, member of the DOE Cosmic Visions Dark Energy 21$\,$cm Working Group, and design and construction
 of the radio telescope used for the 21$\,$cm Baryon Mapping eXperiment at Brookhaven National
 Laboratory.
\item[\textnormal{2013}] \textbf{Gravitational Waves}\\
 Proposed a new method for the indirect detection of gravitational waves via precision 
 stellar redshift measurement.
\item[\textnormal{2012}] \textbf{Modified Newtonian Dynamics}\\
 Investigated the plausibility of Modified Newtonian Dynamics on a local scale based on
 rotation curves of the Milky Way.

\end{description}

\end{rSection}

%----------------------------------------------------------------------------------------
%	REFEREED PUBLICATIONS SECTION
%----------------------------------------------------------------------------------------

\begin{rSection}{Refereed Publications}

\begin{etaremune}
\item {Fabritius}, J.~M., \textbf{{Arena}, E.~J.}, {Goldberg}, D.~M. \textit{``Shape, Color, and Distance in Weak Gravitational Flexion,"} MNRAS 501, 4103 (2021) \href{https://arxiv.org/abs/2006.03506}{[arXiv:2006.03506]}
\end{etaremune}

\end{rSection}

%----------------------------------------------------------------------------------------
%	CONFERENCE PROCEEDINGS, SCIENCE BOOKS, WHITE PAPERS SECTION
%----------------------------------------------------------------------------------------

\begin{rSection}{Conference Proceedings, Science Books, White Papers}

\begin{etaremune}

\item {Ahmed}, Z., {Alonso}, D., {Amin}, M.~A.,
         {Ansari}, R., \textbf{{Arena}, E.~J.}, {Bandura}, K.,
         {Battaglia}, N, {Blazek}, J., 
         {Bull}, P., {Castorina}, E.,
         {Chang}, T.-C., {Connor}, L., {Dav{\'e}}, R., {Dillon}, J.~S.,
         {Dvorkin}, C.,
         {van Engelen}, A., {Ferraro}, S., {Flauger}, R.,
         {Foreman}, S., {Frisch}, J., {Green}, D.,
         {Holder}, G., {Jacobs}, D., {Johnson}, M.~C.,
         {Karagiannis}, D.,
         {Kaurov}, A.~A., {Knox}, L., 
         {Liu}, A., {Loverde}, M., {Ma}, Y.-Z., {Masui}, K.~W.,
         {McClintock}, T., {Meerburg}, P.~D., {Moodley}, K.,
         {M{\"u}nchmeyer}, M., {Newburgh}, L.~B., {Ng}, C.,
         {Nomerotski}, A., {O'Connor}, P., {Obuljen}, A.,
         {Padmanabhan}, H., {Parkinson}, D., {Prochaska}, J.~X.,
         {Rajendran}, S.,
         {Rapetti}, D., {Saliwanchik}, B., {Schaan}, E., {Sehgal}, N.,
         {Shaw}, J.~R., {Sheehy}, C., {Sheldon}, E.,
         {Shirley}, R., {Silverstein}, E., {Slatyer}, T.,
         {Slosar}, A., {Stankus}, P., {Stebbins}, A.,
         {Timbie}, P., {Tucker}, G.~S., {Tyndall}, W.,
         {Villaescusa-Navarro}, F., {Wallisch}, B., and {White}, M.,
\textit{``Packed Ultra-wideband Mapping Array (PUMA): A Radio Telescope for Cosmology and Transients,''} ArXiv e-prints (2019) \href{https://arxiv.org/abs/1907.12559}{[arXiv:1907.12559]}

\item {Ahmed}, Z., {Alonso}, D., {Amin}, M.~A.,
         {Ansari}, R., \textbf{{Arena}, E.~J.}, {Bandura}, K.,
         {Beardsley}, A., {Bull}, P., {Castorina}, E.,
         {Chang}, T.-C., {Dav{\'e}}, R., {Dillon}, J.~S.,
         {van Engelen}, A., {Ewall-Wice}, A., {Ferraro}, S.,
         {Foreman}, S., {Frisch}, J., {Green}, D.,
         {Holder}, G., {Jacobs}, D., {Karagiannis}, D.,
         {Kaurov}, A.~A., {Knox}, L., {Kuhn}, E.,
         {Liu}, A., {Ma}, Y.-Z., {Masui}, K.~W.,
         {McClintock}, T., {Moodley}, K.,
         {M{\"u}nchmeyer}, M., {Newburgh}, L.~B.,
         {Nomerotski}, A., {O'Connor}, P., {Obuljen}, A.,
         {Padmanabhan}, H., {Parkinson}, D., {Perdereau}, O.,
         {Rapetti}, D., {Saliwanchik}, B., {Sehgal}, N.,
         {Shaw}, J.~R., {Sheehy}, C., {Sheldon}, E.,
         {Shirley}, R., {Silverstein}, E., {Slatyer}, T.,
         {Slosar}, A., {Stankus}, P., {Stebbins}, A.,
         {Timbie}, P., {Tucker}, G.~S., {Tyndall}, W.,
         {Villaescusa-Navarro}, F., and {Wulf}, D.,
\textit{``Research and Development for HI Intensity Mapping,''} ArXiv e-prints (2019) \href{https://arxiv.org/abs/1907.13090}{[arXiv:1907.13090]}


\item {Cosmic Visions 21$\,$cm Collaboration}, {Ansari}, R., \textbf{{Arena}, E.~J.} , 
	{Bandura}, K., {Bull}, P., {Castorina}, E., {Chang}, T.-C., 
	{Foreman}, S., {Frisch}, J., {Green}, D., {Karagiannis}, D., 
	{Liu}, A., {Masui}, K.~W., {Meerburg}, P.~D., {Newburgh}, L.~B., 
	{Obuljen}, A., {O'Connor}, P., {Shaw}, J.~R., {Sheehy}, C., 
	{Slosar}, A., {Smith}, K., {Stankus}, P., {Stebbins}, A., 
	{Timbie}, P., {Villaescusa-Navarro}, F., and {White}, M., 
\textit{``Inflation and Early Dark Energy with a {Stage~{\sc ii}} Hydrogen Intensity Mapping experiment,''} ArXiv e-prints (2018) \href{https://arxiv.org/abs/1810.09572}{[arXiv:1810.09572]}

\end{etaremune}

\end{rSection}

%----------------------------------------------------------------------------------------
%	TALKS AND PRESENTATIONS SECTION
%----------------------------------------------------------------------------------------

\begin{rSection}{Conferences and Talks}

\textbf{Contributed Talks}\\
``Hybrid analytic image modeling and image moments approach to gravitational lensing''\\
\forceindent Public talk for my Phyics Ph.D. Candidacy Exam, Drexel University \hfill 4 Jun. 2020\\
\forceindent Research talk to incoming graduate students, Drexel University \hfill 17 Sep. 2019\\
``Observation of gravitational waves through precision stellar redshift measurement''\\
\forceindent High School Research Program conference, Brookhaven National Laboratory \hfill 16 Aug. 2013

\textbf{Poster Presentations}\\
``Hybrid analytic image modeling and image moments approach to gravitational lensing''\\
\forceindent First-year graduate student presentations, Drexel University \hfill 11 Jun. 2019\\
``Dark matter and its alternatives''\\
\forceindent High School Research Program conference, Brookhaven National Laboratory \hfill 27 Nov. 2012

\end{rSection}

%----------------------------------------------------------------------------------------
%	SOFTWARE SECTION
%----------------------------------------------------------------------------------------

\begin{rSection}{Software Developed}

\underline{\makebox[0.965\textwidth][l]{\textbf{Authored}}}

\begin{description}[leftmargin=10em, style=nextline]

\item[\texttt{Lenser}] A tool for measuring weak gravitational flexion. \textit{Publicly available code written in Python}. \href{https://github.com/DrexelLenser/Lenser}{https://github.com/DrexelLenser/Lenser}

\item[21cmMG] A suite for probing modified gravity with 21\,cm cosmology. \textit{Publicly available code written in Python}. \href{https://github.com/evanjarena/21cmMG}{https://github.com/evanjarena/21cmMG}

\item[Fisher21cm] Fisher forecast for a general 21\,cm experiment. \textit{Publicly available code written in Python}. \href{https://github.com/evanjarena/Fisher21cm}{https://github.com/evanjarena/Fisher21cm}

\end{description}

\underline{\makebox[0.965\textwidth][l]{\textbf{Contributed}}}

\begin{description}[leftmargin=10em, style=nextline]

\item[\texttt{LensTools}] Useful computing tools for weak lensing analyses. \textit{Publicly available code written in Python}. \href{https://github.com/apetri/LensTools}{https://github.com/apetri/LensTools}

\end{description}

\end{rSection}

%----------------------------------------------------------------------------------------
%	TEACHING SECTION
%----------------------------------------------------------------------------------------

\begin{rSection}{Teaching}

\textbf{Drexel University} \\ 
\textit{Teaching Assistant} (Recitation and Lab Instructor)\\
\forceindent PHYS 100, \textit{Preparation for Engineering Studies}  \hfill {Winter 2020, Winter 2019}\\
\forceindent PHYS 152, \textit{Introductory Physics I}  \hfill {Spring 2021, Spring 2020, Spring 2019}\\
\forceindent PHYS 154, \textit{Introductory Physics III}  \hfill {Fall 2020, Fall 2019, Fall 2018}\\
\textbf{Stony Brook University} \\
\textit{Lecturer}\\
\forceindent  Della Pietra High School Applied Math Program \hfill {Spring 2017}

\end{rSection}

%----------------------------------------------------------------------------------------
%	PROFESSIONAL ACTIVITIES AND SERVICE SECTION
%----------------------------------------------------------------------------------------

\begin{rSection}{Professional Activities and Service}

\begin{description}[leftmargin=10em, style=nextline]

\item[Working Groups] Inactive member of the DOE Cosmic Visions Dark Energy 21$\,$cm Working 
 Group

\item[Collaborations] Inactive member of the Large Synoptic Survey Telescope Dark Energy 
 Science Collaboration (LSST-DESC)

\end{description}

\textbf{Outreach Activities}\\
Invited to appear on the Drexel University Teaching Assistant Orientation Panel, as part of the Teaching Assistant Orientation and Preparation Course GRAD T580 (17 Sep. 2020).

Gave a physics demonstration at the Kaczmarczik Lecture Series Open House, hosted by the Drexel University Department of Physics (14 Nov. 2018).

\textbf{Committee Work}\\
Treasurer of the Drexel University Physics Graduate Student Association (2020 -- 2021).

\end{rSection}


%----------------------------------------------------------------------------------------
%	TECHNICAL SKILLS SECTION
%----------------------------------------------------------------------------------------

%\begin{rSection}{Technical Skills}
%
%\begin{description}[leftmargin=16em, style=nextline]
%
%\item[Computer Languages] Python, Fortran
%\item[Tools] Mathematica
%\item[Markup] \LaTeX, HTML, CSS

%\end{description}

%\end{rSection}


\end{document}
