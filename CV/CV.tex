%----------------------------------------------------------------------------------------
%       Evan J. Arena
%       Curriculum Vitae
%       Last updated: 17 April 2018
%----------------------------------------------------------------------------------------
%	PACKAGES AND OTHER DOCUMENT CONFIGURATIONS
%----------------------------------------------------------------------------------------

\documentclass{resume} % Use the custom resume.cls style

\usepackage[left=0.75in,top=0.6in,right=0.75in,bottom=0.6in]{geometry} % Document margins
\usepackage{hyperref}
\usepackage[usenames,dvipsnames]{xcolor}
\hypersetup{
    colorlinks = true,
    citecolor = {MidnightBlue},
    linkcolor = {BrickRed},
    urlcolor = {BrickRed}
}
\usepackage{enumitem}

%\usepackage{bibentry}

\newcommand{\forceindent}{\leavevmode{\parindent=1em\indent}}

\name{Evan J. Arena} % Name


\address{Disque Hall, Office No. 705 \\ 32 S. 32$^{\rm nd}$ St. \\ Philadelphia, PA 19104, USA} % Address
\address{+1~$\cdot$~(516)~$\cdot$~383~$\cdot$~4817 \\ \href{mailto:evan.james.arena@drexel.edu}{evan.james.arena@drexel.edu}} % Phone number and email
\address{Ph.D. Student \\ Deptartment of Physics \\ Drexel University } % Title

\pagenumbering{roman}

\begin{document}

  %\nobibliography{CV.bib}
  %\bibliographystyle{unsrt}

%----------------------------------------------------------------------------------------
%	RESEARCH INTERESTS SECTION
%----------------------------------------------------------------------------------------

\begin{rSection}{Research Interests}

\textbf{Theoretical astrophysics and cosmology}, including large-scale structure, 21$\,$cm cosmology, dark energy, inflation, general relativity, modified gravity, gravitational lensing, dark matter, radio astronomy, and gravitational waves. 

\end{rSection}

%----------------------------------------------------------------------------------------
%	EDUCATION SECTION
%----------------------------------------------------------------------------------------

\begin{rSection}{Education}

\textbf{Drexel University} \hfill {2018 -- Present} \\ 
Ph.D. Student of Physics

\textbf{Stony Brook University} \hfill {2013 -- 2017} \\
B.S. in Physics and Astronomy/Planetary Sciences\\
\textit{Cum Laude}\\
\textit{Departmental Honors in Physics}

\end{rSection}

%----------------------------------------------------------------------------------------
%	POSITIONS HELD SECTION
%----------------------------------------------------------------------------------------

\begin{rSection}{Positions Held}

\textbf{Drexel University} \hfill {2018 -- Present} \\
\textit{Graduate Research Assistant and CoAS Dean's Fellow}\\
Department of Physics

\textbf{Stony Brook University and Brookhaven National Laboratory} \hfill {2015 -- Present}\\
\textit{Research Assistant}\\
SBU Department of Physics \& Astronomy  and BNL Department of Physics

\textbf{Brookhaven National Laboratory} \hfill {2012 -- 2013}\\
\textit{Intern}\\
Department of Physics

\end{rSection}

%----------------------------------------------------------------------------------------
%	AWARDS AND HONORS SECTION
%----------------------------------------------------------------------------------------

\begin{rSection}{Awards and Honors}

\textit{College of Arts and Sciences (CoAS) Dean's Fellowship}, Drexel University \hfill {2018}\\
\textit{Sigma Pi Sigma National Physics Honor Society Member}, Stony Brook University \hfill{2017}\\
\textit{Presidential Scholarship}, Stony Brook University \hfill {2013}\\

\end{rSection}

%----------------------------------------------------------------------------------------
%	RESEARCH HISTORY SECTION
%----------------------------------------------------------------------------------------

\begin{rSection}{Research History}

\begin{description}[leftmargin=8em, style=nextline]

\item[\textnormal{2018 -- Present}] \textbf{Gravitational Lensing}\\
Study of the second-order weak gravitational lensing effect known as Flexion. 
\item[\textnormal{2015 -- Present}] \textbf{Low redshift 21$\,$cm intensity mapping}\\
 Cosmological parameter and modified gravity forecasts for a general 21$\,$cm cosmology 
 experiment, member of the DOE Cosmic Visions Dark Energy 21$\,$cm Working Group, and design and construction
 of the radio telescope used for the 21$\,$cm Baryon Mapping eXperiment at Brookhaven National
 Laboratory.
\item[\textnormal{2013}] \textbf{Gravitational Waves}\\
 Proposed a new method for the indirect detection of gravitational waves via precision 
 stellar redshift measurement.
\item[\textnormal{2012}] \textbf{Modified Newtonian Dynamics}\\
 Investigated the plausibility of Modified Newtonian Dynamics on a local scale based on
 rotation curves of the Milky Way.

\end{description}

\end{rSection}

%----------------------------------------------------------------------------------------
%	PROFESSIONAL ACTIVITIES AND SERVICE SECTION
%----------------------------------------------------------------------------------------

\begin{rSection}{Professional Activities and Service}

\begin{description}[leftmargin=10em, style=nextline]

\item[Working Groups] Member of the DOE Cosmic Visions Dark Energy 21$\,$cm Working 
 Group

\item[Collaborations] Member of the Large Synoptic Survey Telescope Dark Energy 
 Science Collaboration (LSST-DESC)

\end{description}

\end{rSection}

%----------------------------------------------------------------------------------------
%	TEACHING SECTION
%----------------------------------------------------------------------------------------

\begin{rSection}{Teaching}

\textbf{Drexel University} \\ 
\textit{Teaching Assistant}\\
\forceindent PHYS 100, \textit{Preparation for Engineering Studies} (Recitation Instructor) \hfill {Winter 2019}
\forceindent PHYS 154, \textit{Introductory Physics III} (Recitation and Lab Instructor) \hfill {Fall 2018}\\



\textbf{Stony Brook University} \\
\textit{Lecturer}\\
\forceindent  Della Pietra High School Applied Math Program \hfill {Spring 2017}

\end{rSection}

%----------------------------------------------------------------------------------------
%	TALKS AND PRESENTATIONS SECTION
%----------------------------------------------------------------------------------------

\begin{rSection}{Talks and Presentations}

``Observation of gravitational waves through precision stellar redshift measurement''\\
\forceindent High School Research Program conference, Brookhaven National Laboratory \hfill August 2013
 
``Dark Matter and its alternatives'' (Poster)\\
\forceindent High School Research Program poster session, Brookhaven National Laboratory \hfill September 2012

\end{rSection}

%----------------------------------------------------------------------------------------
%	PUBLICATIONS SECTION
%----------------------------------------------------------------------------------------

\begin{rSection}{Publications}

{Cosmic Visions 21$\,$cm Collaboration}, {Ansari}, R., \textbf{{Arena}, E.~J.} , 
	{Bandura}, K., {Bull}, P., {Castorina}, E., {Chang}, T.-C., 
	{Foreman}, S., {Frisch}, J., {Green}, D., {Karagiannis}, D., 
	{Liu}, A., {Masui}, K.~W., {Meerburg}, P.~D., {Newburgh}, L.~B., 
	{Obuljen}, A., {O'Connor}, P., {Shaw}, J.~R., {Sheehy}, C., 
	{Slosar}, A., {Smith}, K., {Stankus}, P., {Stebbins}, A., 
	{Timbie}, P., {Villaescusa-Navarro}, F., and {White}, M., 
\textit{``Inflation and Early Dark Energy with a {Stage~{\sc ii}} Hydrogen Intensity Mapping experiment''}, ArXiv e-prints (2018) \href{https://arxiv.org/abs/1810.09572}{[arXiv:1810.09572]}

\end{rSection}



%----------------------------------------------------------------------------------------
%	REFEREED PUBLICATIONS SECTION
%----------------------------------------------------------------------------------------

%\begin{rSection}{Refereed Publications}

%TBA

%\end{rSection}

%----------------------------------------------------------------------------------------
%	CONFERENCE PROCEEDINGS, SCIENCE BOOKS, WHITE PAPERS SECTION
%----------------------------------------------------------------------------------------

%\begin{rSection}{Conference Proceedings, Science Books, White Papers}

%\textbf{E. J. Arena}, et al., \textit{Cosmic Visions Dark Energy: 21-cm Roadmap} (2018), [arXiv:TBA] (In preparation)

%\end{rSection}


%----------------------------------------------------------------------------------------
%	TECHNICAL SKILLS SECTION
%----------------------------------------------------------------------------------------

%\begin{rSection}{Technical Skills}
%
%\begin{description}[leftmargin=16em, style=nextline]
%
%\item[Computer Languages] Python, Fortran
%\item[Tools] Mathematica
%\item[Markup] \LaTeX, HTML, CSS

%\end{description}

%\end{rSection}


\end{document}
